\documentclass[conference,compsoc]{IEEEtran}

\ifCLASSOPTIONcompsoc
  % IEEE Computer Society needs nocompress option
  % requires cite.sty v4.0 or later (November 2003)
  \usepackage[nocompress]{cite}
\else
  % normal IEEE
  \usepackage{cite}
\fi

% \usepackage{arxiv} # uncomment for preprint

\usepackage[utf8]{inputenc} % allow utf-8 input
\usepackage[T1]{fontenc}    % use 8-bit T1 fonts
% \usepackage{hyperref}       % hyperlinks
\usepackage{url}            % simple URL typesetting
\usepackage{booktabs}       % professional-quality tables
\usepackage{amsfonts}       % blackboard math symbols
\usepackage{nicefrac}       % compact symbols for 1/2, etc.
\usepackage{microtype}      % microtypography
\usepackage{lipsum}
\usepackage{graphicx}
% \usepackage{subfigure}
\usepackage{amsmath}
\usepackage{amssymb}
\usepackage{amsthm}
\usepackage{algorithm}
\usepackage{algorithmic}
\usepackage{hyperref}
\usepackage{xr}
\usepackage{subfig}
\usepackage{xcolor}


\DeclareMathOperator*{\argmax}{arg\,max}
\DeclareMathOperator*{\argmin}{arg\,min}
\DeclareMathOperator{\score}{score}

% Note that the amsmath package sets \interdisplaylinepenalty to 10000
% thus preventing page breaks from occurring within multiline equations. Use:
\interdisplaylinepenalty=2500
% after loading amsmath to restore such page breaks as IEEEtran.cls normally
% does. amsmath.sty is already installed on most LaTeX systems. The latest
% version and documentation can be obtained at:
% http://www.ctan.org/pkg/amsmath

\usepackage{placeins}
% \usepackage{algpseudocode}

% \algnewcommand\algorithmicforeach{\textbf{foreach}}
% \algdef{S}[FOR]{ForEach}[1]{\algorithmicforeach\ #1\ \algorithmicdo}

\usepackage{hyperref}

% Attempt to make hyperref and algorithmic work together better:
\newcommand{\theHalgorithm}{\arabic{algorithm}}
\makeatletter
\newcommand{\linebreakand}{%
  \end{@IEEEauthorhalign}
  \hfill\mbox{}\par
  \mbox{}\hfill\begin{@IEEEauthorhalign}
}
\makeatother

\begin{document}
\title{Title}
% author names and affiliations
% use a multiple column layout for up to three different
% affiliations
\author{
\IEEEauthorblockN{Someone}
\IEEEauthorblockA{Computer Science\\
University of Rhode Island\\
Kingston, RI 02881\\
email@gmail.com}
\and
\IEEEauthorblockN{Someone}
\IEEEauthorblockA{Computer Science\\
University of Rhode Island\\
Kingston, RI 02881\\
email@uri.edu}
\and
\IEEEauthorblockN{Someone}
\IEEEauthorblockA{Computer Science \\
University of Rhode Island\\
Kingston, RI 02881\\
email@gmail.com}
\linebreakand
\IEEEauthorblockN{Someone}
\IEEEauthorblockA{Computer Science\\
University of Rhode Island\\
Kingston, RI 02881\\
email@gmail.crom}
\and
\IEEEauthorblockN{Someone}
\IEEEauthorblockA{Computer Science \\
University of Rhode Island\\
Kingston, RI 02881\\
email@uri.edu}}

\IEEEoverridecommandlockouts
\IEEEpubid{\makebox[\columnwidth]{978-1-6654-3902-2/21/\$31.00~\copyright2021 IEEE \hfill} \hspace{\columnsep}\makebox[\columnwidth]{ }}
% make the title area
\maketitle
\IEEEpubidadjcol
% this must go after the closing bracket ] following \twocolumn[ ...

\begin{abstract}
 abstract
\end{abstract}


\IEEEpeerreviewmaketitle
    \section{Introduction}
\label{sec:introduction}


We introduce the project
    \section{Methods}
\label{sec:methods}


We say the methods. 
    \section{Results}
\label{sec:results}

We show the results.
    \section{Discussion}
\label{sec:discussion}

We discusss shit. 


    % \afterpage{\clearpage}
    \FloatBarrier
    \bibliographystyle{IEEEtran}
    \bibliography{references}
\end{document}
